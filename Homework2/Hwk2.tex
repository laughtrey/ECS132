\documentclass{article}
\usepackage{amsmath}
\numberwithin{equation}{section}

%\usepackage{showframe} % This line can be used to clearly show the new margins
\usepackage[margin=1.2in]{geometry}
\newcommand{\eqname}[1]{\tag*{#1}}% Tag equation with name

\begin{document}

\title{	
	\normalfont\normalsize
	\textsc{University of California, Davis} \\ % Your university, school and/or department name(s)
	\vspace{25pt} % Whitespace
	\rule{\linewidth}{0.5pt} \\ % Thin top horizontal rule
	\vspace{20pt} % Whitespace
	{\huge ECS 132 Homework 2} \\ % The assignment title
	\vspace{12pt} % Whitespace
	\rule{\linewidth}{2pt} \\ % Thick bottom horizontal rule
	\vspace{12pt} % Whitespace
}

\author{Rohan Skariah \and Nathan Krieger \and Raymond Laughrey \and Geoffrey Cook}
\date{\normalsize\today}
\maketitle

\begin{abstract}
This files includes the solutions to HW 2 from ECS 132 Winter Quarter. There are 4 attached .R files for problems 2, 3, 5, and 7.
\end{abstract}

\section{Question 1}

For question 1, we are given the following information to find the Var(XYZ).
\\ \indent 
X is a indicator random variable with probability p
\\ \indent 
Y is a indicator random variable with probability q
\\ \indent
Z is a indicator random variable with probability r


\begin{align} 
	    \label{Var(XYZ)}
		\text{Var(XYZ)} &=  E(X^2Y^2Z^2) - [E(XYZ)]^2\eqname{} \\
		&= [E(X^2) \cdot E(Y^2) \cdot E(Z^2)] - [EX \cdot EY \cdot EZ]^2 \eqname{} \\
		&= ([Var(X) + (EX)^2] \cdot [Var(Y) + (EY)^2] \cdot [Var(Z) + (EZ)^2]) - [EX \cdot EY \cdot EZ]^2 \eqname{} \\
		&= ([p(1-p) + p^2] \cdot [q(1-q) + q^2] \cdot [r(1-r) + r^2]) - [p \cdot q \cdot r]^2 \eqname{Using 3.78 and 3.79} \\
		&= ([p - p^2 + p^2] \cdot [q - q^2 + q^2] \cdot [r - r^2 + r^2]) - [p \cdot q \cdot r]^2 \eqname{} \\
		&= (p \cdot q \cdot r) - [p \cdot q \cdot r]^2 \eqname{} \\
		&= pqr - (pqr)^2  = pqr(1-pqr) \eqname{} \\
\end{align}


\section{Question 2}

Please see the file Code2.R

\section{Question 3}

Please see the file Code3.R

\section{Question 4}

For question 4, we are given the following information to find the Var(X).
\\ \indent 
A double geometric distribution family has support of all real integers.
\\ \indent 
The family is indexed by p and q and we are given the P(X=K) for all the possible Ks.

First, we are going to find an equation to calculate Var(X) 
\begin{align} 
	    \label{Var(X) Derived}
		Var(X) &=  E(X^2) - EX^2 \eqname{} \\
		&= E(X^2) - EX + EX - EX^2 \eqname{- EX + EX = + 0} \\
		&= E(X^2 - X) + EX - EX^2 \eqname{Factored into EX} \\
		&= E[X(X-1)] + EX - EX^2 \eqname{Algebra} \\
\end{align}

So, now we will solve for EX
\begin{align} 
	    \label{EX}
		EX &=  \sum_{k = -\infty}^{\infty} k \cdot P(X=k) \eqname{} \\
		&= \sum_{k = -\infty}^{-1} k \cdot P(X=k) + \sum_{k = 0}^{0} k \cdot P(X=k) + \sum_{k = 1}^{\infty} k \cdot P(X=k) \eqname{} \\
		&= (\sum_{i = -\infty}^{-1} k\cdot c (1-p)^{|k|- 1} \cdot p) + (0 \cdot q) + (\sum_{k = 1}^{\infty} k\cdot c (1-p)^{|k|- 1} \cdot p) \eqname{} \\
		&= 2 \cdot cp \cdot \frac{1}{[1-(1-p)]^2} \eqname{} \\
		&= \frac{2cp}{p^2} = \frac{2c}{p} \eqname{} \\
\end{align}


Now, we will solve for E[X(X-1)]
\begin{align} 
	    \label{E[X(X-1)]}
		E[X(X-1)] &=  (c \sum_{X=-\infty}^{-1} X(X-1) \cdot (1-p)^{X-1}\cdot p) + (0 \cdot q) + (c \sum_{X=1}^{\infty}X(X-1) \cdot (1-p)^{X-1} \cdot p) \eqname{} \\
		&= 2 \cdot (c \sum_{X=1}^{\infty}X(X-1) \cdot (1-p)^{X-1} \cdot p) \eqname{} \\
		&= 2 \cdot (c \sum_{X=2}^{\infty}p(p-1) \cdot X(X-1) \cdot (1-p)^{X-2}) \eqname{Factor out (1-p)} \\
    	&= 2 \cdot (c \frac{2p(1-p)}{(1-(1-p))^3}) \eqname{2nd Derivative of Geometric Series} \\
    	&= \frac{4c \cdot (1-p)}{p^2} \eqname{Algebra} \\
\end{align}

Since we know EX and E[X(X-1)], we can solve for Var(X) 
\begin{align} 
	    \label{Var(X)}
		Var(X) &= E[X(X-1)] + EX - EX^2 \eqname{} \\
		&= \frac{4c \cdot (1-p)}{p^2} + \frac{2c}{p} - (\frac{2c}{p})^2 \eqname{} \\
\end{align}

\section{Question 5}

Please see the file Code5.R

\section{Question 6}

For question 1, we are given the following information.
\\ \indent 
X is a random variable with exponential distribution
\\ \indent 
We need to prove that cX will have exponential distribution when c is greater than 0
First, we will find the cumulative distribution function in terms of X
\begin{align} 
	    \label{F_Y(t)}
		\text{F\textsubscript{Y}(t)} &=  P(Y \leq t) \eqname{definition of F\textsubscript{Y}} \\
		&= P(cX \leq t) \eqname{definition of Y} \\
		&= P(X \leq \frac{t}{c}) \eqname{algebra} \\
		&= F\textsubscript{X}(\frac{t}{c}) \eqname{definition of F\textsubscript{X}} \\
\end{align}

Then, to find the probability density function, we do the following
\begin{align} 
	    \label{f_Y(t)}
		\text{f\textsubscript{Y}(t)} &=  \frac{d}{dt} F\textsubscript{Y}(t) \eqname{definition of F\textsubscript{y}} \\
		&= \frac{d}{dt} F\textsubscript{x}(\frac{t}{c}) \eqname{from (6.1)} \\
		&= f\textsubscript{x}(\frac{t}{c}) \cdot \frac{d}{dt}(\frac{t}{c}) \eqname{definition of f\textsubscript{x} and chain rule} \\
		&= \frac{\lambda}{c} e^{-\frac{\lambda}{c} \cdot t} \cdot \eqname{using 7.40} \\
\end{align}
Here, we have the final density in the form of of the exponential family
\begin{equation}
    \lambda e ^{- \lambda t} \text{ where } \lambda = \frac{\lambda}{c}, 0 < t < \infty,\text{and } c > 0
\end{equation}
This proves that if X has exponential distribution, cX does as well if c is greater than 0.

\section{Question 7}

Please see the file Code7.R



\end{document}
